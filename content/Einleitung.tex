\chapter{Überblick der CPU}
\label{chap:Überblick der CPU}

Die Central Processing Unit (CPU) ist die zentrale Komponente eines Computers, die alle Rechenoperationen ausführt und die
Verarbeitung der Signale steuert. Sie lädt Befehle aus dem Arbeitsspeicher (RAM), führt Berechnungen oder logische Operationen
aus und schreibt das Ergebnis zurück in den Speicher. Dadurch fungiert sie als das „Gehirn“ des Computersystems.
\\\\
Die CPU setzt sich aus mehreren wichtigen Komponenten zusammen. Zu den Hauptbestandteilen gehören das \textbf{Steuerwerk}
(Control Unit), das den Datenfluss innerhalb der CPU koordiniert, sowie die \textbf{arithmetisch-logische Einheit} (ALU),
die für mathematische und logische Operationen verantwortlich ist.
\\\\
Weitere essenzielle Komponenten der CPU sind die \textbf{Register}, die temporäre Speicherplätze für Daten,
Operanden und Adressen bereitstellen. \\ Die CPU verwendet fünf solcher Register, die in verschiedenen Modulen integriert sind.
Das \texttt{OpReg} enthält ein Register für die aktuelle Operation (\autoref{TODO}), während sich jeweils zwei Register in den
Modulen \texttt{SWR} (\autoref{TODO}) und \texttt{RegADR} (\autoref{TODO}) befinden.
\\\\
Diese Register haben folgende Aufgaben:

\begin{table}[H]
    \centering
    \begin{tabular}{|l|l|p{8cm}|}
        \hline
        \textbf{Register} & \textbf{Komponente} & \textbf{Beschreibung}                                             \\ \hline
        \texttt{Op}       & \texttt{OpReg}      & Enthält die aktuelle 4-Bit-Operation, die ausgeführt werden soll. \\ \hline
        \texttt{A}        & \texttt{SWR}        & Speichert den ersten 4-Bit-Operanden für die ALU.                 \\ \hline
        \texttt{B}        & \texttt{SWR}        & Speichert den zweiten 4-Bit-Operanden für die ALU.                \\ \hline
        \texttt{ADR1}     & \texttt{RegADR}     & Enthält die 4 MSB (Most Significant Bits) der RAM-Adresse.        \\ \hline
        \texttt{ADR2}     & \texttt{RegADR}     & Enthält die 4 LSB (Least Significant Bits) der RAM-Adresse.       \\ \hline
    \end{tabular}
    \caption{Übersicht der Register und ihrer Komponenten}
    \label{tab:register}
\end{table}

\noindent Das Verständnis der Registerstruktur ist essenziell für das Verständnis der Funktionsweise der CPU, da die Register
den Ablauf der Befehlsverarbeitung und die Speicheradressierung maßgeblich beeinflussen.
