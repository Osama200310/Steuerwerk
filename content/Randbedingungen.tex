\chapter{Randbedingungen}
\label{chap:randbedingungen}

\section{Zustandscodierung}
\label{sec:Zustandscodierung}

\begin{table} [H]
    \centering
    \begin{tabular}{|l|c|c|c|c|}
        \hline
        \multirow{2}{*}{\textbf{Zustand}} & \multicolumn{4}{c}{\textbf{Codierung}}                \\

                                          & Q3                                     & Q2 & Q1 & Q0 \\
        \hline
        \textbf{Fetch}                    & 0                                      & 0  & 0  & 0  \\
        \hline
        \textbf{SET\_PC}                  & 0                                      & 0  & 1  & 0  \\
        \hline
        \textbf{ALU1}                     & 0                                      & 1  & 0  & 0  \\
        \hline
        \textbf{ALU2}                     & 0                                      & 1  & 0  & 1  \\
        \hline
        \textbf{JMXA1}                    & 1                                      & 0  & 0  & 0  \\
        \hline
        \textbf{JMXA2}                    & 1                                      & 0  & 0  & 1  \\
        \hline
        \textbf{LDA}                      & 1                                      & 1  & 0  & 0  \\
        \hline
        \textbf{SVA}                      & 1                                      & 1  & 0  & 1  \\
        \hline
        \textbf{SWR}                      & 1                                      & 1  & 1  & 0  \\
        \hline
        \textbf{END}                      & 1                                      & 1  & 1  & 1  \\
        \hline
        \textbf{Don´t Care}               & -                                      & -  & -  & -  \\
        \hline
    \end{tabular}
    \caption{Zustandscodierung des Steuerwerks}
    \label{tab:Zustandscodierung}
\end{table}