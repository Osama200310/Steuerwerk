\chapter{Anforderungen}
\label{chap:anforderungen}

Das Steuerwerk soll folgende Funktionen ausführen können:


\begin{table}[ht]
    \centering
    \begin{tabular}{| l | l | l | l| l |}
        \hline
        \cellcolor[gray]{0.8} \textbf{Berechnungsart} & \cellcolor[gray]{0.8} \textbf{So} & \cellcolor[gray]{0.8} \textbf{S1} &
        \cellcolor[gray]{0.8} \textbf{S2}             & \cellcolor[gray]{0.8} \textbf{S3}                                             \\
        \hline
        And                                           & 0                                 & 0                                 & 0 & 0 \\
        \hline
        Or                                            & 0                                 & 0                                 & 0 & 1 \\
        \hline
        Not                                           & 0                                 & 0                                 & 1 & 0 \\
        \hline
        Add                                           & 0                                 & 0                                 & 1 & 1 \\
        \hline
        Sub                                           & 0                                 & 1                                 & 0 & 0 \\
        \hline
        Mul                                           & 0                                 & 1                                 & 0 & 1 \\
        \hline
        Reserved1                                     & 0                                 & 1                                 & 1 & 0 \\
        \hline
        Reserved2                                     & 0                                 & 1                                 & 1 & 1 \\
        \hline
        JMP                                           & 1                                 & 0                                 & 0 & 0 \\
        \hline
        JMPC                                          & 1                                 & 0                                 & 0 & 1 \\
        \hline
        JMPO                                          & 1                                 & 0                                 & 1 & 0 \\
        \hline
        JMPZ                                          & 1                                 & 0                                 & 1 & 1 \\
        \hline
        LDA                                           & 1                                 & 1                                 & 0 & 0 \\
        \hline
        SVA                                           & 1                                 & 1                                 & 0 & 1 \\
        \hline
        SWR                                           & 1                                 & 1                                 & 1 & 0 \\
        \hline
        END                                           & 1                                 & 1                                 & 1 & 1 \\
        \hline
    \end{tabular}
    \caption{Berechnungsart Operationen}
    \label{tab:berechnungsart-operationen}
\end{table}

\begin{itemize}
    \item Jump (JMP): An eine beliebige Adresse springen, unbedingter Sprung
    \item Jump Carry (JMPC): Wenn ein Carry ausgegeben wird, springt man an die gegebene Adresse, bedingter Sprung
    \item Jump Overflow (JMPO): Wenn ein Overflow erzeugt wird, dann wird an gegebene Adresse gesprungen, bedingter Sprung
    \item Jump Zero (JMPZ): Wenn der Wert leer ist, wird an diese Adresse gesprungen, bedingter Sprung
    \item Load A (LDA): ein Wert des RAM wird in Register A geladen
    \item Save A (SVA): Wert der in A steht wird in den RAM gespeichert
    \item Switch Register (SWR): Inhalte von Register A und B tauschen
    \item End (END): Rechnenoperation wird beendet
\end{itemize}